\subsection[Orakelmaschinen]{Orakelmaschinen}

\begin{frame}
	\begin{Satz}[Baker,Gill,Solovay, 75]
		Es existieren Orakel A, B so dass ${\P}^A = {\NP}^A$ und ${\P}^B \neq {\NP}^B$
	\end{Satz}
\end{frame}

\begin{frame}{Probleme für $\P \neq \NP$}
	\begin{itemize}[<+->]
	\item Ein Beweis der Diagonalisierung verwendet gilt auch für TM mit Orakel!
	\item $\Rightarrow$ ein Beweis für die P-NP Frage kann keine Diagonalisierung verwenden!
	\end{itemize}
\end{frame}

\begin{frame}{Beweis : ${\P}^A = {\NP}^A$ }
	\begin{itemize}[<+->]
	
	\item ${\P}^A = {\NP}^A$ haben wir gerade schon gesehen:
			Nutze einfach das Orakel A = $\mathbf{EXPCOM}$
	\item B zu konstruieren ist schwieriger
	\end{itemize}	
\end{frame}

\begin{frame}{Beweis : ${\P}^B \neq {\NP}^B$}
	\begin{Definition}
		Für eine Sprache B sei $U_B = \lbrace 1^n :$ Es gibt einen String
		der L\"ange n in B $\rbrace $
	\end{Definition}	
	\pause
	
	\begin{itemize}[<+->]
		\item Wir sehen sofort ein : $U_B \in {\NP}^B$ , da eine nicht det. TM
			ein Zertifikat raten kann.
		\item M\"ussen also nur noch B so konstruieren, dass $U_B \notin {\P}^B$
	\end{itemize}
\end{frame}

\begin{frame}{Konstruktion von B}
	Wir konstruieren eine Folge von Sprachen $(B_i)_{i \in \mathbb{N}}$ so , dass 
	$B = \lim_{n \to \infty} B_i$	
	\begin{itemize}[<+->]
		\item Wie stellen wir sicher, dass alle Turing Maschinen $U_B$ nicht
			in polynomieller Zeit entscheiden können?
		\item Tipp: Die Menge aller Turing Maschinen ist abzählbar
	\end{itemize}
\end{frame}

\begin{frame}{Konstruktion von B}
	\begin{itemize}[<+->]
	\item Genau : iterieren über alle Turing Maschinen $M_i$ und stellen sicher, dass
		$M_i$ nicht in polynomieller Zeit $U_B$ entscheiden kann
	\item Nutze dabei, dass die Anzahl der Wörter exponentiell in der Eingabelänge wächst
	\end{itemize}
\end{frame}

\begin{frame}{Konstruktion von $B_i$}
	Wir fangen an mit $B_0 = \emptyset$. Konstruktion f\"r $B_i$ :
	\begin{itemize}
		\item Wähle n so , dass n größer als alle Strings in $B_{i-1}$
		\item Lasse $M_i$ auf Eingabe $1^n$ genau $2^n / 10$ Schritte laufen
	\end{itemize}
\end{frame}