\begin{frame}
	\frametitle{Grenzen der Diagonalisierung}
	\framesubtitle{Wiederholung Diagonalisierung}
	\begin{KITinfoblock}{Was ist Diagonalisierung} {
			Als Diagonalisierung wird (hier) ein Beweis bezeichnet, der nur auf den beiden folgenden
			Eigenschaften von TM aufbaut.
			
			\begin{enumerate}
				\item<2-> Die Existenz einer Repräsentation von TM durch Zeichenketten (Gödelnummer)
				\item<3-> Die Fähigkeit eine andere TM mit geringem zusätzlichen Zeit- oder Platzbedarf zu simulieren (Universelle TM)
			\end{enumerate}		
		}
	\end{KITinfoblock}
\end{frame}

\begin{frame}
	\frametitle{Grenzen der Diagonalisierung}
	\framesubtitle{Definition von Orakelmschinen}
	\begin{KITinfoblock}{Definiton Orakel-Turingmaschine} {
			Eine Orakel-Turingmaschine $M$ ist eine TM, die folgende zusätzliche Eigenschaften hat:
			\begin{itemize}
				\item<2-> ein spezielles zusätzliches Band (Orakelband) und 3 spezielle zusätzliche Zustände $q_{query}, q_{yes}, q_{no}$.
				\item <3-> ein Orakel $O \subset \{0,1\}^*$
				\item <4-> Wenn $M$ den Zustand $q_{query}$ betritt, ist der Folgezustand
					\begin{itemize}
					\item $q_{yes}$, wenn für Inhalt $s$ des Orakelbands gilt $s \in O$ und
				    \item	$q_{no}$, wenn  $s \notin O$ 
					\end{itemize} 
				\item<5-> Das Orakel liefert die Antwort in einem Berechnungsschritt
			\end{itemize}
		}
	\end{KITinfoblock}
\end{frame}