\begin{frame}
	\frametitle{Einleitung}
	\framesubtitle{Diagonalisierung : was ist das eigentlich?}
	
	\includegraphics[scale = 0.3]{halting-problem.jpg}
\end{frame}
\begin{frame}
	\frametitle{Einleitung}
	\framesubtitle{Eine Hierarchie von Komplexitätsklassen}
	\includegraphics[scale=0.5]{images/timehierarchy.pdf}
\end{frame}
\begin{frame}
	\frametitle{Einleitung}
	\framesubtitle{$\P$ oder $\NPC$ : gibt es noch mehr in $\NP$?}
	\overprint{
		\only<1>{\includegraphics[page = 1,scale = 0.6]{images/npi.pdf}}
		\only<2>{\includegraphics[page = 2,scale = 0.6]{images/npi.pdf}}
	}
\end{frame}
\begin{frame}
	\frametitle{Einleitung}
	\framesubtitle{Grenzen der Diagonalisierung}
	
	\begin{itemize}[<+->]
	  \item Orakelmaschinen und die P, NP Frage
	  \item Polynomial Hierarchy : Eine Verallgemeinerung von \P , \NP
	\end{itemize}
\end{frame}
