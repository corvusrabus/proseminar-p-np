\begin{frame}
	\frametitle{Diagonalisierung}
	\framesubtitle{Was verstehen wir darunter?}
	
	
	

			\includegraphics[scale = 0.6]{images/Halteproblem.pdf}%
%	\begin{figure}
		%	\begin{minipage}{0.7\linewidth}
				\begin{itemize}[<+->]
					\item Unentscheidbarkeit des Halteproblems
					\item Diagonalisierung nicht immer "schön" zu sehen
					\item In späteren Beweisen gewisse Abstraktion
				\end{itemize}
		%	\end{minipage}
		%	\begin{minipage}{0.3\linewidth}
		
		%			\includegraphics[scale = 0.4]{images/Halteproblem.pdf}%
		%	\end{minipage}
	%	\end{figure}
	
\end{frame}
\begin{frame}
	\frametitle{Diagonalisierung}
	\framesubtitle{Was verstehen wir darunter?}
	\begin{KITinfoblock}{Was ist Diagonalisierung} {
			Als Diagonalisierung wird (in der Informatik) ein Beweis bezeichnet, der nur auf den beiden folgenden
			Eigenschaften von TM aufbaut.
			
			\begin{enumerate}
				\item<2-> Die Existenz einer Repräsentation von TM durch Zeichenketten (Gödelnummer)
				\item<3-> Die Fähigkeit eine andere TM mit geringem zusätzlichen Zeit- oder Platzbedarf zu simulieren (Universelle TM)
			\end{enumerate}		
		}
	\end{KITinfoblock}
	
\end{frame}