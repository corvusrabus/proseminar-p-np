\begin{frame}
	hier muss noch die Definition von $\PH$ rein ;)
\end{frame}
\begin{frame} 
	\frametitle{Die Klasse $\PH$}
	\framesubtitle{Eigenschaften von $\PH$}
	
	\begin{itemize}[<+->]
		\item Vermutung: $\P \neq \NP$ und $\NP \neq \coNP$	
		\item Verallgemeinerung:    $\sum_{i}^{p}  \subsetneq \sum_{i+1}^{p}$ für alle $i$
		\item "The polynomial hierarchy does not collapse"
	\end{itemize}
	\bigskip
	\pause
	\begin{KITinfoblock}{Satz Kollaps von $\PH$ und Auswirkungen auf $\P-\NP$}
		\begin{enumerate}[<+->]
			\item Für alle $ i \geq 0$ gilt: $ \quad \sum_{i}^{p} = \prod_{i}^{p} \quad \Rightarrow \quad \PH = \sum_{i}^{p}$
			\item Wenn $\P = \NP$, dann folgt $\PH = \P$
		\end{enumerate}
	\end{KITinfoblock}
\end{frame}
\begin{frame} 
	\frametitle{Die Klasse $\PH$}
	\framesubtitle{Beweis von 2.}
	\heading{Beweis von $ \P = \NP \Rightarrow \PH = \P$}
	\begin{itemize}[<+->]
		\item Sei $\P = \NP$, beweisen über Induktion $\sum_{i}^{p},\prod_{i}^{p} \subset \P$
	\end{itemize}
\end{frame} 