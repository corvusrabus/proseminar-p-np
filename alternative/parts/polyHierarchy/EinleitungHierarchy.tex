\begin{frame}
		\frametitle{Motivation und Beispiele}
		\framesubtitle{Verallgemeinerung}
		
		\begin{itemize}[<+->]
			\item bisher die Komplexitätsklassen $\P, \NP , \coNP$
			\item es gibt Probleme, die sich nicht mit diesen klassifizieren lassen
			\item durch Verallgemeinerung dieser Klassen kann eine Reihe weiterer Probleme "eingefangen" werden
			\item Verallgemeinerung ist die "polynomielle Hierarchie" $\PH$ 
		\end{itemize}
		\pause
		\includegraphics[scale = 0.4]{images/polyhierarchy.pdf}
\end{frame}
\begin{frame}
	\frametitle{Motivation und Beispiele}
	\framesubtitle{Beispiele}
	
	\begin{KITinfoblock}{Definition $\mathbf{INDSET}$}
	Sei $\mathbf{INDSET} = \lbrace \langle G,k \rangle $ : Graph $G$ hat ein
	Independent set, welches Größe k hat $\rbrace$
	\end{KITinfoblock}
	\bigskip
	\pause
	Bekannt : $\mathbf{INDSET} \in \NPC$
	\bigskip
	\pause
	\begin{KITinfoblock}{Definition $\mathbf{EXACT INDSET}$}
	Sei $\mathbf{EXACT INDSET} = \lbrace \langle G,k \rangle $ : das größte
	independent set in G hat Größe genau k$\rbrace$ \newline
	=$\lbrace \langle G,k \rangle$ : $\exists$ independent set der Größe k in $G$
	und $\forall$ independent sets in $G$ haben Größe $\leq k \rbrace$
	\end{KITinfoblock}
\end{frame}
\begin{frame}
	\frametitle{Motivation und Beispiele}
	\framesubtitle{Die Klasse $\sum_{2}^{p} $}
	
	\begin{overprint}
	\begin{KITinfoblock}{$\mathbf{\only<2>{EXACT}INDSET}$}
	Sei $\mathbf{INDSET} = \lbrace \langle G,k \rangle $ : $\exists$ independent
	set in G, welches Größe k hat \only<2>{und $\forall$ independent sets in $G$
	haben Größe $\leq k$}$\rbrace$
	\end{KITinfoblock}
	\end{overprint}
	
	\bigskip
	\begin{overprint}
	

	\begin{KITinfoblock}{\only<1>{Wiederholung \NP} \only<2>{Definition
	$\sum_{2}^{p} $ }}
		\only<1>{ \NP} \only<2>{$\sum_{2}^{p} $ }
		ist die Menge aller Sprachen L f\"ur die gilt : \newline
		Es gibt eine deterministische polynomielle TM $M$ und ein Polynom $q$ so, dass
		:
		\newline
		
		$x \in L \Leftrightarrow \exists u \in {\lbrace 0,1 \rbrace }^{q(|x|)}$
		\only<2>{$\forall v \in {\lbrace 0,1 \rbrace }^{q(|x|)}$}
		$M(x,u \only<2>{,v}) = 1$
	\end{KITinfoblock}
	\end{overprint}
	
\end{frame}
\begin{frame}
	\frametitle{Motivation und Beispiele}
	\framesubtitle{Noch mehr Quantoren?}
	\begin{center}
		\includegraphics[scale = 0.15]{images/more.jpg}
	\end{center}
\end{frame}